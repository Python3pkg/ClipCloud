%
% API Documentation for API Documentation
% Module lib.pyjson
%
% Generated by epydoc 3.0.1
% [Thu Apr 26 18:39:20 2012]
%

%%%%%%%%%%%%%%%%%%%%%%%%%%%%%%%%%%%%%%%%%%%%%%%%%%%%%%%%%%%%%%%%%%%%%%%%%%%
%%                          Module Description                           %%
%%%%%%%%%%%%%%%%%%%%%%%%%%%%%%%%%%%%%%%%%%%%%%%%%%%%%%%%%%%%%%%%%%%%%%%%%%%

    \index{lib \textit{(package)}!lib.pyjson \textit{(module)}|(}
\section{Module lib.pyjson}

    \label{lib:pyjson}

%%%%%%%%%%%%%%%%%%%%%%%%%%%%%%%%%%%%%%%%%%%%%%%%%%%%%%%%%%%%%%%%%%%%%%%%%%%
%%                               Functions                               %%
%%%%%%%%%%%%%%%%%%%%%%%%%%%%%%%%%%%%%%%%%%%%%%%%%%%%%%%%%%%%%%%%%%%%%%%%%%%

  \subsection{Functions}

    \label{lib:pyjson:parse_json}
    \index{lib \textit{(package)}!lib.pyjson \textit{(module)}!lib.pyjson.parse\_json \textit{(function)}}

    \vspace{0.5ex}

\hspace{.8\funcindent}\begin{boxedminipage}{\funcwidth}

    \raggedright \textbf{parse\_json}(\textit{s})

\setlength{\parskip}{2ex}
\setlength{\parskip}{1ex}
    \end{boxedminipage}

    \label{lib:pyjson:save_json}
    \index{lib \textit{(package)}!lib.pyjson \textit{(module)}!lib.pyjson.save\_json \textit{(function)}}

    \vspace{0.5ex}

\hspace{.8\funcindent}\begin{boxedminipage}{\funcwidth}

    \raggedright \textbf{save\_json}(\textit{d})

\setlength{\parskip}{2ex}
\setlength{\parskip}{1ex}
    \end{boxedminipage}


%%%%%%%%%%%%%%%%%%%%%%%%%%%%%%%%%%%%%%%%%%%%%%%%%%%%%%%%%%%%%%%%%%%%%%%%%%%
%%                               Variables                               %%
%%%%%%%%%%%%%%%%%%%%%%%%%%%%%%%%%%%%%%%%%%%%%%%%%%%%%%%%%%%%%%%%%%%%%%%%%%%

  \subsection{Variables}

    \vspace{-1cm}
\hspace{\varindent}\begin{longtable}{|p{\varnamewidth}|p{\vardescrwidth}|l}
\cline{1-2}
\cline{1-2} \centering \textbf{Name} & \centering \textbf{Description}& \\
\cline{1-2}
\endhead\cline{1-2}\multicolumn{3}{r}{\small\textit{continued on next page}}\\\endfoot\cline{1-2}
\endlastfoot\raggedright \_\-\_\-p\-a\-c\-k\-a\-g\-e\-\_\-\_\- & \raggedright \textbf{Value:} 
{\tt \texttt{'}\texttt{lib}\texttt{'}}&\\
\cline{1-2}
\end{longtable}


%%%%%%%%%%%%%%%%%%%%%%%%%%%%%%%%%%%%%%%%%%%%%%%%%%%%%%%%%%%%%%%%%%%%%%%%%%%
%%                           Class Description                           %%
%%%%%%%%%%%%%%%%%%%%%%%%%%%%%%%%%%%%%%%%%%%%%%%%%%%%%%%%%%%%%%%%%%%%%%%%%%%

    \index{lib \textit{(package)}!lib.pyjson \textit{(module)}!lib.pyjson.PyJson \textit{(class)}|(}
\subsection{Class PyJson}

    \label{lib:pyjson:PyJson}
Class to abstract the process of reading, writing and parsing JSON files 
using Python's inbuilt json module.


%%%%%%%%%%%%%%%%%%%%%%%%%%%%%%%%%%%%%%%%%%%%%%%%%%%%%%%%%%%%%%%%%%%%%%%%%%%
%%                                Methods                                %%
%%%%%%%%%%%%%%%%%%%%%%%%%%%%%%%%%%%%%%%%%%%%%%%%%%%%%%%%%%%%%%%%%%%%%%%%%%%

  \subsubsection{Methods}

    \label{lib:pyjson:PyJson:__init__}
    \index{lib \textit{(package)}!lib.pyjson \textit{(module)}!lib.pyjson.PyJson \textit{(class)}!lib.pyjson.PyJson.\_\_init\_\_ \textit{(method)}}

    \vspace{0.5ex}

\hspace{.8\funcindent}\begin{boxedminipage}{\funcwidth}

    \raggedright \textbf{\_\_init\_\_}(\textit{self}, \textit{path}, \textit{base}={\tt \texttt{\{}\texttt{\}}})

    \vspace{-1.5ex}

    \rule{\textwidth}{0.5\fboxrule}
\setlength{\parskip}{2ex}
\begin{alltt}

Create the file if it does not exist and add an empty JSON object to it

Arguments:
- path: The path to the file to store the JSON in
- base: The root structure of the JSON document
\end{alltt}

\setlength{\parskip}{1ex}
    \end{boxedminipage}

    \label{lib:pyjson:PyJson:add}
    \index{lib \textit{(package)}!lib.pyjson \textit{(module)}!lib.pyjson.PyJson \textit{(class)}!lib.pyjson.PyJson.add \textit{(method)}}

    \vspace{0.5ex}

\hspace{.8\funcindent}\begin{boxedminipage}{\funcwidth}

    \raggedright \textbf{add}(\textit{self}, \textit{key}, \textit{value})

    \vspace{-1.5ex}

    \rule{\textwidth}{0.5\fboxrule}
\setlength{\parskip}{2ex}
    Add a record to the Python dictionary

\setlength{\parskip}{1ex}
    \end{boxedminipage}

    \label{lib:pyjson:PyJson:remove}
    \index{lib \textit{(package)}!lib.pyjson \textit{(module)}!lib.pyjson.PyJson \textit{(class)}!lib.pyjson.PyJson.remove \textit{(method)}}

    \vspace{0.5ex}

\hspace{.8\funcindent}\begin{boxedminipage}{\funcwidth}

    \raggedright \textbf{remove}(\textit{self}, \textit{key})

    \vspace{-1.5ex}

    \rule{\textwidth}{0.5\fboxrule}
\setlength{\parskip}{2ex}
    Remove a record from the Python dictionary

\setlength{\parskip}{1ex}
    \end{boxedminipage}

    \label{lib:pyjson:PyJson:save}
    \index{lib \textit{(package)}!lib.pyjson \textit{(module)}!lib.pyjson.PyJson \textit{(class)}!lib.pyjson.PyJson.save \textit{(method)}}

    \vspace{0.5ex}

\hspace{.8\funcindent}\begin{boxedminipage}{\funcwidth}

    \raggedright \textbf{save}(\textit{self})

    \vspace{-1.5ex}

    \rule{\textwidth}{0.5\fboxrule}
\setlength{\parskip}{2ex}
    Save a json representation of the Python dictionary to the file

\setlength{\parskip}{1ex}
    \end{boxedminipage}

    \index{lib \textit{(package)}!lib.pyjson \textit{(module)}!lib.pyjson.PyJson \textit{(class)}|)}
    \index{lib \textit{(package)}!lib.pyjson \textit{(module)}|)}
