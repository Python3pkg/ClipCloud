%
% API Documentation for API Documentation
% Module lib.dropbox
%
% Generated by epydoc 3.0.1
% [Thu Apr 26 18:39:20 2012]
%

%%%%%%%%%%%%%%%%%%%%%%%%%%%%%%%%%%%%%%%%%%%%%%%%%%%%%%%%%%%%%%%%%%%%%%%%%%%
%%                          Module Description                           %%
%%%%%%%%%%%%%%%%%%%%%%%%%%%%%%%%%%%%%%%%%%%%%%%%%%%%%%%%%%%%%%%%%%%%%%%%%%%

    \index{lib \textit{(package)}!lib.dropbox \textit{(module)}|(}
\section{Module lib.dropbox}

    \label{lib:dropbox}
The Dropbox API and several methods to abstract common operations


%%%%%%%%%%%%%%%%%%%%%%%%%%%%%%%%%%%%%%%%%%%%%%%%%%%%%%%%%%%%%%%%%%%%%%%%%%%
%%                               Functions                               %%
%%%%%%%%%%%%%%%%%%%%%%%%%%%%%%%%%%%%%%%%%%%%%%%%%%%%%%%%%%%%%%%%%%%%%%%%%%%

  \subsection{Functions}

    \label{lib:dropbox:format_path}
    \index{lib \textit{(package)}!lib.dropbox \textit{(module)}!lib.dropbox.format\_path \textit{(function)}}

    \vspace{0.5ex}

\hspace{.8\funcindent}\begin{boxedminipage}{\funcwidth}

    \raggedright \textbf{format\_path}(\textit{path})

    \vspace{-1.5ex}

    \rule{\textwidth}{0.5\fboxrule}
\setlength{\parskip}{2ex}
    Normalize path for use with the Dropbox API.

    This function turns multiple adjacent slashes into single slashes, then
    ensures that there's a leading slash but not a trailing slash.

\setlength{\parskip}{1ex}
    \end{boxedminipage}

    \label{lib:dropbox:match_hostname}
    \index{lib \textit{(package)}!lib.dropbox \textit{(module)}!lib.dropbox.match\_hostname \textit{(function)}}

    \vspace{0.5ex}

\hspace{.8\funcindent}\begin{boxedminipage}{\funcwidth}

    \raggedright \textbf{match\_hostname}(\textit{cert}, \textit{hostname})

    \vspace{-1.5ex}

    \rule{\textwidth}{0.5\fboxrule}
\setlength{\parskip}{2ex}
    Verify that *cert* (in decoded format as returned by 
    SSLSocket.getpeercert()) matches the *hostname*.  RFC 2818 rules are 
    mostly followed, but IP addresses are not accepted for *hostname*.

    CertificateError is raised on failure. On success, the function returns
    nothing.

\setlength{\parskip}{1ex}
    \end{boxedminipage}

    \label{lib:dropbox:create_connection}
    \index{lib \textit{(package)}!lib.dropbox \textit{(module)}!lib.dropbox.create\_connection \textit{(function)}}

    \vspace{0.5ex}

\hspace{.8\funcindent}\begin{boxedminipage}{\funcwidth}

    \raggedright \textbf{create\_connection}(\textit{address})

\setlength{\parskip}{2ex}
\setlength{\parskip}{1ex}
    \end{boxedminipage}


%%%%%%%%%%%%%%%%%%%%%%%%%%%%%%%%%%%%%%%%%%%%%%%%%%%%%%%%%%%%%%%%%%%%%%%%%%%
%%                               Variables                               %%
%%%%%%%%%%%%%%%%%%%%%%%%%%%%%%%%%%%%%%%%%%%%%%%%%%%%%%%%%%%%%%%%%%%%%%%%%%%

  \subsection{Variables}

    \vspace{-1cm}
\hspace{\varindent}\begin{longtable}{|p{\varnamewidth}|p{\vardescrwidth}|l}
\cline{1-2}
\cline{1-2} \centering \textbf{Name} & \centering \textbf{Description}& \\
\cline{1-2}
\endhead\cline{1-2}\multicolumn{3}{r}{\small\textit{continued on next page}}\\\endfoot\cline{1-2}
\endlastfoot\raggedright S\-D\-K\-\_\-V\-E\-R\-S\-I\-O\-N\- & \raggedright \textbf{Value:} 
{\tt \texttt{'}\texttt{1.3}\texttt{'}}&\\
\cline{1-2}
\raggedright T\-R\-U\-S\-T\-E\-D\-\_\-C\-E\-R\-T\-\_\-F\-I\-L\-E\- & \raggedright \textbf{Value:} 
{\tt \texttt{'}\texttt{E:{\textbackslash}{\textbackslash}Dropbox{\textbackslash}{\textbackslash}Code{\textbackslash}{\textbackslash}ClipCloud{\textbackslash}{\textbackslash}Main App{\textbackslash}{\textbackslash}lib{\textbackslash}{\textbackslash}trusted-cer}\texttt{...}}&\\
\cline{1-2}
\raggedright A\-P\-P\-\_\-N\-A\-M\-E\- & \raggedright \textbf{Value:} 
{\tt \texttt{'}\texttt{ClipCloud}\texttt{'}}&\\
\cline{1-2}
\raggedright A\-P\-P\-\_\-P\-A\-T\-H\- & \raggedright \textbf{Value:} 
{\tt \texttt{'}\texttt{C:{\textbackslash}{\textbackslash}Users{\textbackslash}{\textbackslash}Giles{\textbackslash}{\textbackslash}AppData{\textbackslash}{\textbackslash}Roaming{\textbackslash}{\textbackslash}ClipCloud}\texttt{'}}&\\
\cline{1-2}
\raggedright D\-E\-B\-U\-G\- & \raggedright \textbf{Value:} 
{\tt True}&\\
\cline{1-2}
\raggedright G\-I\-T\-H\-U\-B\-\_\-T\-O\-K\-E\-N\-\_\-P\-A\-T\-H\- & \raggedright \textbf{Value:} 
{\tt \texttt{'}\texttt{C:{\textbackslash}{\textbackslash}Users{\textbackslash}{\textbackslash}Giles{\textbackslash}{\textbackslash}AppData{\textbackslash}{\textbackslash}Roaming{\textbackslash}{\textbackslash}ClipCloud{\textbackslash}{\textbackslash}github\_to}\texttt{...}}&\\
\cline{1-2}
\raggedright H\-E\-L\-P\-\_\-M\-E\-S\-S\-A\-G\-E\- & \raggedright \textbf{Value:} 
{\tt \texttt{'}\texttt{{\textbackslash}nClipCloud is a program for easily sharing all kinds of}\texttt{...}}&\\
\cline{1-2}
\raggedright H\-I\-S\-T\-O\-R\-Y\-\_\-P\-A\-T\-H\- & \raggedright \textbf{Value:} 
{\tt \texttt{'}\texttt{C:{\textbackslash}{\textbackslash}Users{\textbackslash}{\textbackslash}Giles{\textbackslash}{\textbackslash}AppData{\textbackslash}{\textbackslash}Roaming{\textbackslash}{\textbackslash}ClipCloud{\textbackslash}{\textbackslash}history.j}\texttt{...}}&\\
\cline{1-2}
\raggedright P\-L\-A\-T\-F\-O\-R\-M\- & \raggedright \textbf{Value:} 
{\tt \texttt{'}\texttt{Windows}\texttt{'}}&\\
\cline{1-2}
\raggedright S\-C\-R\-E\-E\-N\-S\-H\-O\-T\-\_\-P\-A\-T\-H\- & \raggedright \textbf{Value:} 
{\tt \texttt{'}\texttt{C:{\textbackslash}{\textbackslash}Users{\textbackslash}{\textbackslash}Giles{\textbackslash}{\textbackslash}AppData{\textbackslash}{\textbackslash}Roaming{\textbackslash}{\textbackslash}ClipCloud{\textbackslash}{\textbackslash}img}\texttt{'}}&\\
\cline{1-2}
\raggedright S\-H\-A\-R\-E\-\_\-M\-E\-S\-S\-A\-G\-E\- & \raggedright \textbf{Value:} 
{\tt \texttt{'}\texttt{I just uploaded a file to ClipCloud - check it out.}\texttt{'}}&\\
\cline{1-2}
\raggedright S\-H\-A\-R\-I\-N\-G\-\_\-S\-E\-R\-V\-I\-C\-E\-S\- & \raggedright \textbf{Value:} 
{\tt \texttt{[}\texttt{'}\texttt{clipboard}\texttt{'}\texttt{, }\texttt{'}\texttt{facebook}\texttt{'}\texttt{, }\texttt{'}\texttt{twitter}\texttt{'}\texttt{, }\texttt{'}\texttt{email}\texttt{'}\texttt{, }\texttt{'}\texttt{stdout}\texttt{'}\texttt{]}}&\\
\cline{1-2}
\raggedright T\-I\-M\-E\-R\-\_\-A\-C\-T\-I\-V\-A\-T\-E\-D\- & \raggedright \textbf{Value:} 
{\tt True}&\\
\cline{1-2}
\raggedright T\-M\-P\-\_\-P\-A\-T\-H\- & \raggedright \textbf{Value:} 
{\tt \texttt{'}\texttt{C:{\textbackslash}{\textbackslash}Users{\textbackslash}{\textbackslash}Giles{\textbackslash}{\textbackslash}AppData{\textbackslash}{\textbackslash}Roaming{\textbackslash}{\textbackslash}ClipCloud{\textbackslash}{\textbackslash}tmp}\texttt{'}}&\\
\cline{1-2}
\raggedright T\-O\-K\-E\-N\-\_\-P\-A\-T\-H\- & \raggedright \textbf{Value:} 
{\tt \texttt{'}\texttt{C:{\textbackslash}{\textbackslash}Users{\textbackslash}{\textbackslash}Giles{\textbackslash}{\textbackslash}AppData{\textbackslash}{\textbackslash}Roaming{\textbackslash}{\textbackslash}ClipCloud{\textbackslash}{\textbackslash}token.json}\texttt{'}}&\\
\cline{1-2}
\raggedright U\-R\-L\-S\- & \raggedright \textbf{Value:} 
{\tt \texttt{\{}\texttt{'}\texttt{email}\texttt{'}\texttt{: }\texttt{'}\texttt{mailto:?subject=\%s\&body=\%s}\texttt{'}\texttt{, }\texttt{'}\texttt{facebook}\texttt{'}\texttt{: }\texttt{'}\texttt{http}\texttt{...}}&\\
\cline{1-2}
\raggedright \_\-\_\-p\-a\-c\-k\-a\-g\-e\-\_\-\_\- & \raggedright \textbf{Value:} 
{\tt \texttt{'}\texttt{lib}\texttt{'}}&\\
\cline{1-2}
\end{longtable}


%%%%%%%%%%%%%%%%%%%%%%%%%%%%%%%%%%%%%%%%%%%%%%%%%%%%%%%%%%%%%%%%%%%%%%%%%%%
%%                           Class Description                           %%
%%%%%%%%%%%%%%%%%%%%%%%%%%%%%%%%%%%%%%%%%%%%%%%%%%%%%%%%%%%%%%%%%%%%%%%%%%%

    \index{lib \textit{(package)}!lib.dropbox \textit{(module)}!lib.dropbox.Dropbox \textit{(class)}|(}
\subsection{Class Dropbox}

    \label{lib:dropbox:Dropbox}
Methods for interacting with the Dropbox API


%%%%%%%%%%%%%%%%%%%%%%%%%%%%%%%%%%%%%%%%%%%%%%%%%%%%%%%%%%%%%%%%%%%%%%%%%%%
%%                                Methods                                %%
%%%%%%%%%%%%%%%%%%%%%%%%%%%%%%%%%%%%%%%%%%%%%%%%%%%%%%%%%%%%%%%%%%%%%%%%%%%

  \subsubsection{Methods}

    \label{lib:dropbox:Dropbox:__init__}
    \index{lib \textit{(package)}!lib.dropbox \textit{(module)}!lib.dropbox.Dropbox \textit{(class)}!lib.dropbox.Dropbox.\_\_init\_\_ \textit{(method)}}

    \vspace{0.5ex}

\hspace{.8\funcindent}\begin{boxedminipage}{\funcwidth}

    \raggedright \textbf{\_\_init\_\_}(\textit{self}, \textit{in\_user\_mode}={\tt True})

    \vspace{-1.5ex}

    \rule{\textwidth}{0.5\fboxrule}
\setlength{\parskip}{2ex}
    Connect to the Dropbox servers so that files can be uploaded

\setlength{\parskip}{1ex}
    \end{boxedminipage}

    \label{lib:dropbox:Dropbox:upload}
    \index{lib \textit{(package)}!lib.dropbox \textit{(module)}!lib.dropbox.Dropbox \textit{(class)}!lib.dropbox.Dropbox.upload \textit{(method)}}

    \vspace{0.5ex}

\hspace{.8\funcindent}\begin{boxedminipage}{\funcwidth}

    \raggedright \textbf{upload}(\textit{self}, \textit{path}, \textit{filepath}={\tt None})

    \vspace{-1.5ex}

    \rule{\textwidth}{0.5\fboxrule}
\setlength{\parskip}{2ex}
\begin{alltt}

Upload a file to the Dropbox servers

Arguments:
- path: The path to the local copy of the file to be uploaded
- filename: The path, including the filename given to the remote copy of the file
    once it is uploaded to Dropbox. If omitted it defaults to being the same as path
\end{alltt}

\setlength{\parskip}{1ex}
    \end{boxedminipage}

    \label{lib:dropbox:Dropbox:upload_folder}
    \index{lib \textit{(package)}!lib.dropbox \textit{(module)}!lib.dropbox.Dropbox \textit{(class)}!lib.dropbox.Dropbox.upload\_folder \textit{(method)}}

    \vspace{0.5ex}

\hspace{.8\funcindent}\begin{boxedminipage}{\funcwidth}

    \raggedright \textbf{upload\_folder}(\textit{self}, \textit{folder})

    \vspace{-1.5ex}

    \rule{\textwidth}{0.5\fboxrule}
\setlength{\parskip}{2ex}
\begin{alltt}

Upload a folder to the Dropbox servers

Arguments:
- folder: A string representing the path to the folder to be uploaded
\end{alltt}

\setlength{\parskip}{1ex}
    \end{boxedminipage}

    \label{lib:dropbox:Dropbox:get_link}
    \index{lib \textit{(package)}!lib.dropbox \textit{(module)}!lib.dropbox.Dropbox \textit{(class)}!lib.dropbox.Dropbox.get\_link \textit{(method)}}

    \vspace{0.5ex}

\hspace{.8\funcindent}\begin{boxedminipage}{\funcwidth}

    \raggedright \textbf{get\_link}(\textit{self}, \textit{filename})

    \vspace{-1.5ex}

    \rule{\textwidth}{0.5\fboxrule}
\setlength{\parskip}{2ex}
\begin{alltt}

Get the URL of the copy of a file or folder hosted on Dropbox

Arguments:
- filename: The path to the file or folder within the user's Dropbox to get the URL of

Returns: A string containing the URL, in the format http://db.tt/xxxxxxxx
\end{alltt}

\setlength{\parskip}{1ex}
    \end{boxedminipage}

    \label{lib:dropbox:Dropbox:create_folder}
    \index{lib \textit{(package)}!lib.dropbox \textit{(module)}!lib.dropbox.Dropbox \textit{(class)}!lib.dropbox.Dropbox.create\_folder \textit{(method)}}

    \vspace{0.5ex}

\hspace{.8\funcindent}\begin{boxedminipage}{\funcwidth}

    \raggedright \textbf{create\_folder}(\textit{self}, \textit{folder\_name}, \textit{num}={\tt 1})

    \vspace{-1.5ex}

    \rule{\textwidth}{0.5\fboxrule}
\setlength{\parskip}{2ex}
\begin{alltt}

Create a folder in the user's Dropbox

Arguments:
- folder\_name: The name of the folder to create
- num: The number to append to the folder name if a folder with the specified name already exists
\end{alltt}

\setlength{\parskip}{1ex}
    \end{boxedminipage}


%%%%%%%%%%%%%%%%%%%%%%%%%%%%%%%%%%%%%%%%%%%%%%%%%%%%%%%%%%%%%%%%%%%%%%%%%%%
%%                            Class Variables                            %%
%%%%%%%%%%%%%%%%%%%%%%%%%%%%%%%%%%%%%%%%%%%%%%%%%%%%%%%%%%%%%%%%%%%%%%%%%%%

  \subsubsection{Class Variables}

    \vspace{-1cm}
\hspace{\varindent}\begin{longtable}{|p{\varnamewidth}|p{\vardescrwidth}|l}
\cline{1-2}
\cline{1-2} \centering \textbf{Name} & \centering \textbf{Description}& \\
\cline{1-2}
\endhead\cline{1-2}\multicolumn{3}{r}{\small\textit{continued on next page}}\\\endfoot\cline{1-2}
\endlastfoot\raggedright A\-C\-C\-E\-S\-S\-\_\-T\-Y\-P\-E\- & \raggedright \textbf{Value:} 
{\tt \texttt{'}\texttt{app\_folder}\texttt{'}}&\\
\cline{1-2}
\end{longtable}

    \index{lib \textit{(package)}!lib.dropbox \textit{(module)}!lib.dropbox.Dropbox \textit{(class)}|)}

%%%%%%%%%%%%%%%%%%%%%%%%%%%%%%%%%%%%%%%%%%%%%%%%%%%%%%%%%%%%%%%%%%%%%%%%%%%
%%                           Class Description                           %%
%%%%%%%%%%%%%%%%%%%%%%%%%%%%%%%%%%%%%%%%%%%%%%%%%%%%%%%%%%%%%%%%%%%%%%%%%%%

    \index{lib \textit{(package)}!lib.dropbox \textit{(module)}!lib.dropbox.DropboxClient \textit{(class)}|(}
\subsection{Class DropboxClient}

    \label{lib:dropbox:DropboxClient}
\begin{tabular}{cccccc}
% Line for object, linespec=[False]
\multicolumn{2}{r}{\settowidth{\BCL}{object}\multirow{2}{\BCL}{object}}
&&
  \\\cline{3-3}
  &&\multicolumn{1}{c|}{}
&&
  \\
&&\multicolumn{2}{l}{\textbf{lib.dropbox.DropboxClient}}
\end{tabular}

The main access point of doing REST calls on Dropbox. You should first 
create and configure a dropbox.session.DropboxSession object, and then pass
it into DropboxClient's constructor. DropboxClient then does all the work 
of properly calling each API method with the correct OAuth authentication.

You should be aware that any of these methods can raise a 
rest.ErrorResponse exception if the server returns a non-200 or invalid 
HTTP response. Note that a 401 return status at any point indicates that 
the user needs to be reauthenticated.


%%%%%%%%%%%%%%%%%%%%%%%%%%%%%%%%%%%%%%%%%%%%%%%%%%%%%%%%%%%%%%%%%%%%%%%%%%%
%%                                Methods                                %%
%%%%%%%%%%%%%%%%%%%%%%%%%%%%%%%%%%%%%%%%%%%%%%%%%%%%%%%%%%%%%%%%%%%%%%%%%%%

  \subsubsection{Methods}

    \vspace{0.5ex}

\hspace{.8\funcindent}\begin{boxedminipage}{\funcwidth}

    \raggedright \textbf{\_\_init\_\_}(\textit{self}, \textit{session}, \textit{proxy\_host}={\tt None}, \textit{proxy\_port}={\tt None})

\setlength{\parskip}{2ex}
    x.\_\_init\_\_(...) initializes x; see help(type(x)) for signature

\setlength{\parskip}{1ex}
      Overrides: object.\_\_init\_\_ 	extit{(inherited documentation)}

    \end{boxedminipage}

    \label{lib:dropbox:DropboxClient:request}
    \index{lib \textit{(package)}!lib.dropbox \textit{(module)}!lib.dropbox.DropboxClient \textit{(class)}!lib.dropbox.DropboxClient.request \textit{(method)}}

    \vspace{0.5ex}

\hspace{.8\funcindent}\begin{boxedminipage}{\funcwidth}

    \raggedright \textbf{request}(\textit{self}, \textit{target}, \textit{params}={\tt None}, \textit{method}={\tt \texttt{'}\texttt{POST}\texttt{'}}, \textit{content\_server}={\tt False})

    \vspace{-1.5ex}

    \rule{\textwidth}{0.5\fboxrule}
\setlength{\parskip}{2ex}
\begin{alltt}
Make an HTTP request to a target API method.

This is an internal method used to properly craft the url, headers, and
params for a Dropbox API request.  It is exposed for you in case you
need craft other API calls not in this library or if you want to debug it.

Args:
    target: The target URL with leading slash (e.g. '/files')
    params: A dictionary of parameters to add to the request
    method: An HTTP method (e.g. 'GET' or 'POST')
    content\_server: A boolean indicating whether the request is to the
       API content server, for example to fetch the contents of a file
       rather than its metadata.

Returns:
    A tuple of (url, params, headers) that should be used to make the request.
    OAuth authentication information will be added as needed within these fields.
\end{alltt}

\setlength{\parskip}{1ex}
    \end{boxedminipage}

    \label{lib:dropbox:DropboxClient:account_info}
    \index{lib \textit{(package)}!lib.dropbox \textit{(module)}!lib.dropbox.DropboxClient \textit{(class)}!lib.dropbox.DropboxClient.account\_info \textit{(method)}}

    \vspace{0.5ex}

\hspace{.8\funcindent}\begin{boxedminipage}{\funcwidth}

    \raggedright \textbf{account\_info}(\textit{self})

    \vspace{-1.5ex}

    \rule{\textwidth}{0.5\fboxrule}
\setlength{\parskip}{2ex}
\begin{alltt}
Retrieve information about the user's account.

Returns:
    A dictionary containing account information.

    For a detailed description of what this call returns, visit:
    https://www.dropbox.com/developers/docs\#account-info
\end{alltt}

\setlength{\parskip}{1ex}
    \end{boxedminipage}

    \label{lib:dropbox:DropboxClient:put_file}
    \index{lib \textit{(package)}!lib.dropbox \textit{(module)}!lib.dropbox.DropboxClient \textit{(class)}!lib.dropbox.DropboxClient.put\_file \textit{(method)}}

    \vspace{0.5ex}

\hspace{.8\funcindent}\begin{boxedminipage}{\funcwidth}

    \raggedright \textbf{put\_file}(\textit{self}, \textit{full\_path}, \textit{file\_obj}, \textit{overwrite}={\tt False}, \textit{parent\_rev}={\tt None})

    \vspace{-1.5ex}

    \rule{\textwidth}{0.5\fboxrule}
\setlength{\parskip}{2ex}
\begin{alltt}
Upload a file.

Args:
    full\_path: The full path to upload the file to, *including the file name*.
        If the destination directory does not yet exist, it will be created.
    file\_obj: A file-like object to upload. If you would like, you can pass a string as file\_obj.
    overwrite: Whether to overwrite an existing file at the given path. [default False]
        If overwrite is False and a file already exists there, Dropbox
        will rename the upload to make sure it doesn't overwrite anything.
        You need to check the metadata returned for the new name.
        This field should only be True if your intent is to potentially
        clobber changes to a file that you don't know about.
    parent\_rev: The rev field from the 'parent' of this upload. [optional]
        If your intent is to update the file at the given path, you should
        pass the parent\_rev parameter set to the rev value from the most recent
        metadata you have of the existing file at that path. If the server
        has a more recent version of the file at the specified path, it will
        automatically rename your uploaded file, spinning off a conflict.
        Using this parameter effectively causes the overwrite parameter to be ignored.
        The file will always be overwritten if you send the most-recent parent\_rev,
        and it will never be overwritten if you send a less-recent one.

Returns:
    A dictionary containing the metadata of the newly uploaded file.

    For a detailed description of what this call returns, visit:
    https://www.dropbox.com/developers/docs\#files-put

Raises:
    A dropbox.rest.ErrorResponse with an HTTP status of
       400: Bad request (may be due to many things; check e.error for details)
       503: User over quota

Note: In Python versions below version 2.6, httplib doesn't handle file-like objects.
    In that case, this code will read the entire file into memory (!).
\end{alltt}

\setlength{\parskip}{1ex}
    \end{boxedminipage}

    \label{lib:dropbox:DropboxClient:get_file}
    \index{lib \textit{(package)}!lib.dropbox \textit{(module)}!lib.dropbox.DropboxClient \textit{(class)}!lib.dropbox.DropboxClient.get\_file \textit{(method)}}

    \vspace{0.5ex}

\hspace{.8\funcindent}\begin{boxedminipage}{\funcwidth}

    \raggedright \textbf{get\_file}(\textit{self}, \textit{from\_path}, \textit{rev}={\tt None})

    \vspace{-1.5ex}

    \rule{\textwidth}{0.5\fboxrule}
\setlength{\parskip}{2ex}
\begin{alltt}
Download a file.

Unlike most other calls, get\_file returns a raw HTTPResponse with the connection open.
You should call .read() and perform any processing you need, then close the HTTPResponse.

Args:
    from\_path: The path to the file to be downloaded.
    rev: A previous rev value of the file to be downloaded. [optional]

Returns:
    An httplib.HTTPResponse that is the result of the request.

Raises:
    A dropbox.rest.ErrorResponse with an HTTP status of
       400: Bad request (may be due to many things; check e.error for details)
       404: No file was found at the given path, or the file that was there was deleted.
       200: Request was okay but response was malformed in some way.
\end{alltt}

\setlength{\parskip}{1ex}
    \end{boxedminipage}

    \label{lib:dropbox:DropboxClient:get_file_and_metadata}
    \index{lib \textit{(package)}!lib.dropbox \textit{(module)}!lib.dropbox.DropboxClient \textit{(class)}!lib.dropbox.DropboxClient.get\_file\_and\_metadata \textit{(method)}}

    \vspace{0.5ex}

\hspace{.8\funcindent}\begin{boxedminipage}{\funcwidth}

    \raggedright \textbf{get\_file\_and\_metadata}(\textit{self}, \textit{from\_path}, \textit{rev}={\tt None})

    \vspace{-1.5ex}

    \rule{\textwidth}{0.5\fboxrule}
\setlength{\parskip}{2ex}
\begin{alltt}
Download a file alongwith its metadata.

Acts as a thin wrapper around get\_file() (see get\_file() comments for
more details)

Args:
    from\_path: The path to the file to be downloaded.
    rev: A previous rev value of the file to be downloaded. [optional]

Returns:
    - An httplib.HTTPResponse that is the result of the request.
    - A dictionary containing the metadata of the file (see
      https://www.dropbox.com/developers/docs\#metadata for details).

Raises:
    A dropbox.rest.ErrorResponse with an HTTP status of
       400: Bad request (may be due to many things; check e.error for details)
       404: No file was found at the given path, or the file that was there was deleted.
       200: Request was okay but response was malformed in some way.
\end{alltt}

\setlength{\parskip}{1ex}
    \end{boxedminipage}

    \label{lib:dropbox:DropboxClient:file_copy}
    \index{lib \textit{(package)}!lib.dropbox \textit{(module)}!lib.dropbox.DropboxClient \textit{(class)}!lib.dropbox.DropboxClient.file\_copy \textit{(method)}}

    \vspace{0.5ex}

\hspace{.8\funcindent}\begin{boxedminipage}{\funcwidth}

    \raggedright \textbf{file\_copy}(\textit{self}, \textit{from\_path}, \textit{to\_path})

    \vspace{-1.5ex}

    \rule{\textwidth}{0.5\fboxrule}
\setlength{\parskip}{2ex}
\begin{alltt}
Copy a file or folder to a new location.

Args:
    from\_path: The path to the file or folder to be copied.

    to\_path: The destination path of the file or folder to be copied.
        This parameter should include the destination filename (e.g.
        from\_path: '/test.txt', to\_path: '/dir/test.txt'). If there's
        already a file at the to\_path, this copy will be renamed to
        be unique.

Returns:
    A dictionary containing the metadata of the new copy of the file or folder.

    For a detailed description of what this call returns, visit:
    https://www.dropbox.com/developers/docs\#fileops-copy

Raises:
    A dropbox.rest.ErrorResponse with an HTTP status of:

    - 400: Bad request (may be due to many things; check e.error for details)
    - 404: No file was found at given from\_path.
    - 503: User over storage quota.
\end{alltt}

\setlength{\parskip}{1ex}
    \end{boxedminipage}

    \label{lib:dropbox:DropboxClient:file_create_folder}
    \index{lib \textit{(package)}!lib.dropbox \textit{(module)}!lib.dropbox.DropboxClient \textit{(class)}!lib.dropbox.DropboxClient.file\_create\_folder \textit{(method)}}

    \vspace{0.5ex}

\hspace{.8\funcindent}\begin{boxedminipage}{\funcwidth}

    \raggedright \textbf{file\_create\_folder}(\textit{self}, \textit{path})

    \vspace{-1.5ex}

    \rule{\textwidth}{0.5\fboxrule}
\setlength{\parskip}{2ex}
\begin{alltt}
Create a folder.

Args:
    path: The path of the new folder.

Returns:
    A dictionary containing the metadata of the newly created folder.

    For a detailed description of what this call returns, visit:
    https://www.dropbox.com/developers/docs\#fileops-create-folder

Raises:
    A dropbox.rest.ErrorResponse with an HTTP status of
       400: Bad request (may be due to many things; check e.error for details)
       403: A folder at that path already exists.
\end{alltt}

\setlength{\parskip}{1ex}
    \end{boxedminipage}

    \label{lib:dropbox:DropboxClient:file_delete}
    \index{lib \textit{(package)}!lib.dropbox \textit{(module)}!lib.dropbox.DropboxClient \textit{(class)}!lib.dropbox.DropboxClient.file\_delete \textit{(method)}}

    \vspace{0.5ex}

\hspace{.8\funcindent}\begin{boxedminipage}{\funcwidth}

    \raggedright \textbf{file\_delete}(\textit{self}, \textit{path})

    \vspace{-1.5ex}

    \rule{\textwidth}{0.5\fboxrule}
\setlength{\parskip}{2ex}
\begin{alltt}
Delete a file or folder.

Args:
    path: The path of the file or folder.

Returns:
    A dictionary containing the metadata of the just deleted file.

    For a detailed description of what this call returns, visit:
    https://www.dropbox.com/developers/docs\#fileops-delete

Raises:
    A dropbox.rest.ErrorResponse with an HTTP status of

    - 400: Bad request (may be due to many things; check e.error for details)
    - 404: No file was found at the given path.
\end{alltt}

\setlength{\parskip}{1ex}
    \end{boxedminipage}

    \label{lib:dropbox:DropboxClient:file_move}
    \index{lib \textit{(package)}!lib.dropbox \textit{(module)}!lib.dropbox.DropboxClient \textit{(class)}!lib.dropbox.DropboxClient.file\_move \textit{(method)}}

    \vspace{0.5ex}

\hspace{.8\funcindent}\begin{boxedminipage}{\funcwidth}

    \raggedright \textbf{file\_move}(\textit{self}, \textit{from\_path}, \textit{to\_path})

    \vspace{-1.5ex}

    \rule{\textwidth}{0.5\fboxrule}
\setlength{\parskip}{2ex}
\begin{alltt}
Move a file or folder to a new location.

Args:
    from\_path: The path to the file or folder to be moved.
    to\_path: The destination path of the file or folder to be moved.
    This parameter should include the destination filename (e.g.
    from\_path: '/test.txt', to\_path: '/dir/test.txt'). If there's
    already a file at the to\_path, this file or folder will be renamed to
    be unique.

Returns:
    A dictionary containing the metadata of the new copy of the file or folder.

    For a detailed description of what this call returns, visit:
    https://www.dropbox.com/developers/docs\#fileops-move

Raises:
    A dropbox.rest.ErrorResponse with an HTTP status of

    - 400: Bad request (may be due to many things; check e.error for details)
    - 404: No file was found at given from\_path.
    - 503: User over storage quota.
\end{alltt}

\setlength{\parskip}{1ex}
    \end{boxedminipage}

    \label{lib:dropbox:DropboxClient:metadata}
    \index{lib \textit{(package)}!lib.dropbox \textit{(module)}!lib.dropbox.DropboxClient \textit{(class)}!lib.dropbox.DropboxClient.metadata \textit{(method)}}

    \vspace{0.5ex}

\hspace{.8\funcindent}\begin{boxedminipage}{\funcwidth}

    \raggedright \textbf{metadata}(\textit{self}, \textit{path}, \textit{list}={\tt True}, \textit{file\_limit}={\tt 10000}, \textit{hash}={\tt None}, \textit{rev}={\tt None}, \textit{include\_deleted}={\tt False})

    \vspace{-1.5ex}

    \rule{\textwidth}{0.5\fboxrule}
\setlength{\parskip}{2ex}
\begin{alltt}
Retrieve metadata for a file or folder.

Args:
    path: The path to the file or folder.

    list: Whether to list all contained files (only applies when
        path refers to a folder).
    file\_limit: The maximum number of file entries to return within
        a folder. If the number of files in the directory exceeds this
        limit, an exception is raised. The server will return at max
        10,000 files within a folder.
    hash: Every directory listing has a hash parameter attached that
        can then be passed back into this function later to save on                bandwidth. Rather than returning an unchanged folder's contents,                the server will instead return a 304.            rev: The revision of the file to retrieve the metadata for. [optional]
        This parameter only applies for files. If omitted, you'll receive
        the most recent revision metadata.

Returns:
    A dictionary containing the metadata of the file or folder
    (and contained files if appropriate).

    For a detailed description of what this call returns, visit:
    https://www.dropbox.com/developers/docs\#metadata

Raises:
    A dropbox.rest.ErrorResponse with an HTTP status of

    - 304: Current directory hash matches hash parameters, so contents are unchanged.
    - 400: Bad request (may be due to many things; check e.error for details)
    - 404: No file was found at given path.
    - 406: Too many file entries to return.
\end{alltt}

\setlength{\parskip}{1ex}
    \end{boxedminipage}

    \label{lib:dropbox:DropboxClient:thumbnail}
    \index{lib \textit{(package)}!lib.dropbox \textit{(module)}!lib.dropbox.DropboxClient \textit{(class)}!lib.dropbox.DropboxClient.thumbnail \textit{(method)}}

    \vspace{0.5ex}

\hspace{.8\funcindent}\begin{boxedminipage}{\funcwidth}

    \raggedright \textbf{thumbnail}(\textit{self}, \textit{from\_path}, \textit{size}={\tt \texttt{'}\texttt{large}\texttt{'}}, \textit{format}={\tt \texttt{'}\texttt{JPEG}\texttt{'}})

    \vspace{-1.5ex}

    \rule{\textwidth}{0.5\fboxrule}
\setlength{\parskip}{2ex}
\begin{alltt}
Download a thumbnail for an image.

Unlike most other calls, thumbnail returns a raw HTTPResponse with the connection open.
You should call .read() and perform any processing you need, then close the HTTPResponse.

Args:
    from\_path: The path to the file to be thumbnailed.
    size: A string describing the desired thumbnail size.
       At this time, 'small', 'medium', and 'large' are
       officially supported sizes (32x32, 64x64, and 128x128
       respectively), though others may be available. Check
       https://www.dropbox.com/developers/docs\#thumbnails for
       more details.

Returns:
    An httplib.HTTPResponse that is the result of the request.

Raises:
    A dropbox.rest.ErrorResponse with an HTTP status of

    - 400: Bad request (may be due to many things; check e.error for details)
    - 404: No file was found at the given from\_path, or files of that type cannot be thumbnailed.
    - 415: Image is invalid and cannot be thumbnailed.
\end{alltt}

\setlength{\parskip}{1ex}
    \end{boxedminipage}

    \label{lib:dropbox:DropboxClient:thumbnail_and_metadata}
    \index{lib \textit{(package)}!lib.dropbox \textit{(module)}!lib.dropbox.DropboxClient \textit{(class)}!lib.dropbox.DropboxClient.thumbnail\_and\_metadata \textit{(method)}}

    \vspace{0.5ex}

\hspace{.8\funcindent}\begin{boxedminipage}{\funcwidth}

    \raggedright \textbf{thumbnail\_and\_metadata}(\textit{self}, \textit{from\_path}, \textit{size}={\tt \texttt{'}\texttt{large}\texttt{'}}, \textit{format}={\tt \texttt{'}\texttt{JPEG}\texttt{'}})

    \vspace{-1.5ex}

    \rule{\textwidth}{0.5\fboxrule}
\setlength{\parskip}{2ex}
\begin{alltt}
Download a thumbnail for an image alongwith its metadata.

Acts as a thin wrapper around thumbnail() (see thumbnail() comments for
more details)

Args:
    from\_path: The path to the file to be thumbnailed.
    size: A string describing the desired thumbnail size. See thumbnail()
       for details.

Returns:
    - An httplib.HTTPResponse that is the result of the request.
    - A dictionary containing the metadata of the file whose thumbnail
      was downloaded (see https://www.dropbox.com/developers/docs\#metadata
      for details).

Raises:
    A dropbox.rest.ErrorResponse with an HTTP status of

    - 400: Bad request (may be due to many things; check e.error for details)
    - 404: No file was found at the given from\_path, or files of that type cannot be thumbnailed.
    - 415: Image is invalid and cannot be thumbnailed.
    - 200: Request was okay but response was malformed in some way.
\end{alltt}

\setlength{\parskip}{1ex}
    \end{boxedminipage}

    \label{lib:dropbox:DropboxClient:search}
    \index{lib \textit{(package)}!lib.dropbox \textit{(module)}!lib.dropbox.DropboxClient \textit{(class)}!lib.dropbox.DropboxClient.search \textit{(method)}}

    \vspace{0.5ex}

\hspace{.8\funcindent}\begin{boxedminipage}{\funcwidth}

    \raggedright \textbf{search}(\textit{self}, \textit{path}, \textit{query}, \textit{file\_limit}={\tt 1000}, \textit{include\_deleted}={\tt False})

    \vspace{-1.5ex}

    \rule{\textwidth}{0.5\fboxrule}
\setlength{\parskip}{2ex}
\begin{alltt}
Search directory for filenames matching query.

Args:
    path: The directory to search within.

    query: The query to search on (minimum 3 characters).

    file\_limit: The maximum number of file entries to return within a folder.
       The server will return at max 1,000 files.

    include\_deleted: Whether to include deleted files in search results.

Returns:
    A list of the metadata of all matching files (up to
    file\_limit entries).  For a detailed description of what
    this call returns, visit:
    https://www.dropbox.com/developers/docs\#search

Raises:
    A dropbox.rest.ErrorResponse with an HTTP status of
    400: Bad request (may be due to many things; check e.error
    for details)
\end{alltt}

\setlength{\parskip}{1ex}
    \end{boxedminipage}

    \label{lib:dropbox:DropboxClient:revisions}
    \index{lib \textit{(package)}!lib.dropbox \textit{(module)}!lib.dropbox.DropboxClient \textit{(class)}!lib.dropbox.DropboxClient.revisions \textit{(method)}}

    \vspace{0.5ex}

\hspace{.8\funcindent}\begin{boxedminipage}{\funcwidth}

    \raggedright \textbf{revisions}(\textit{self}, \textit{path}, \textit{rev\_limit}={\tt 1000})

    \vspace{-1.5ex}

    \rule{\textwidth}{0.5\fboxrule}
\setlength{\parskip}{2ex}
\begin{alltt}
Retrieve revisions of a file.

Args:
    path: The file to fetch revisions for. Note that revisions
        are not available for folders.
    rev\_limit: The maximum number of file entries to return within
        a folder. The server will return at max 1,000 revisions.

Returns:
    A list of the metadata of all matching files (up to rev\_limit entries).

    For a detailed description of what this call returns, visit:
    https://www.dropbox.com/developers/docs\#revisions

Raises:
    A dropbox.rest.ErrorResponse with an HTTP status of

    - 400: Bad request (may be due to many things; check e.error for details)
    - 404: No revisions were found at the given path.
\end{alltt}

\setlength{\parskip}{1ex}
    \end{boxedminipage}

    \label{lib:dropbox:DropboxClient:restore}
    \index{lib \textit{(package)}!lib.dropbox \textit{(module)}!lib.dropbox.DropboxClient \textit{(class)}!lib.dropbox.DropboxClient.restore \textit{(method)}}

    \vspace{0.5ex}

\hspace{.8\funcindent}\begin{boxedminipage}{\funcwidth}

    \raggedright \textbf{restore}(\textit{self}, \textit{path}, \textit{rev})

    \vspace{-1.5ex}

    \rule{\textwidth}{0.5\fboxrule}
\setlength{\parskip}{2ex}
\begin{alltt}
Restore a file to a previous revision.

Args:
    path: The file to restore. Note that folders can't be restored.
    rev: A previous rev value of the file to be restored to.

Returns:
    A dictionary containing the metadata of the newly restored file.

    For a detailed description of what this call returns, visit:
    https://www.dropbox.com/developers/docs\#restore

Raises:
    A dropbox.rest.ErrorResponse with an HTTP status of

    - 400: Bad request (may be due to many things; check e.error for details)
    - 404: Unable to find the file at the given revision.
\end{alltt}

\setlength{\parskip}{1ex}
    \end{boxedminipage}

    \label{lib:dropbox:DropboxClient:media}
    \index{lib \textit{(package)}!lib.dropbox \textit{(module)}!lib.dropbox.DropboxClient \textit{(class)}!lib.dropbox.DropboxClient.media \textit{(method)}}

    \vspace{0.5ex}

\hspace{.8\funcindent}\begin{boxedminipage}{\funcwidth}

    \raggedright \textbf{media}(\textit{self}, \textit{path})

    \vspace{-1.5ex}

    \rule{\textwidth}{0.5\fboxrule}
\setlength{\parskip}{2ex}
\begin{alltt}
Get a temporary unauthenticated URL for a media file.

All of Dropbox's API methods require OAuth, which may cause problems in
situations where an application expects to be able to hit a URL multiple times
(for example, a media player seeking around a video file). This method
creates a time-limited URL that can be accessed without any authentication,
and returns that to you, along with an expiration time.

Args:
    path: The file to return a URL for. Folders are not supported.

Returns:
    A dictionary that looks like the following example:

    ``\{'url': 'https://dl.dropbox.com/0/view/wvxv1fw6on24qw7/file.mov', 'expires': 'Thu, 16 Sep 2011 01:01:25 +0000'\}``

    For a detailed description of what this call returns, visit:
    https://www.dropbox.com/developers/docs\#media

Raises:
    A dropbox.rest.ErrorResponse with an HTTP status of

    - 400: Bad request (may be due to many things; check e.error for details)
    - 404: Unable to find the file at the given path.
\end{alltt}

\setlength{\parskip}{1ex}
    \end{boxedminipage}

    \label{lib:dropbox:DropboxClient:share}
    \index{lib \textit{(package)}!lib.dropbox \textit{(module)}!lib.dropbox.DropboxClient \textit{(class)}!lib.dropbox.DropboxClient.share \textit{(method)}}

    \vspace{0.5ex}

\hspace{.8\funcindent}\begin{boxedminipage}{\funcwidth}

    \raggedright \textbf{share}(\textit{self}, \textit{path})

    \vspace{-1.5ex}

    \rule{\textwidth}{0.5\fboxrule}
\setlength{\parskip}{2ex}
\begin{alltt}
Create a shareable link to a file or folder.

Shareable links created on Dropbox are time-limited, but don't require any
authentication, so they can be given out freely. The time limit should allow
at least a day of shareability, though users have the ability to disable
a link from their account if they like.

Args:
    path: The file or folder to share.

Returns:
    A dictionary that looks like the following example:

    ``\{'url': 'http://www.dropbox.com/s/m/a2mbDa2', 'expires': 'Thu, 16 Sep 2011 01:01:25 +0000'\}``

    For a detailed description of what this call returns, visit:
    https://www.dropbox.com/developers/docs\#share

Raises:
    A dropbox.rest.ErrorResponse with an HTTP status of

    - 400: Bad request (may be due to many things; check e.error for details)
    - 404: Unable to find the file at the given path.
\end{alltt}

\setlength{\parskip}{1ex}
    \end{boxedminipage}


\large{\textbf{\textit{Inherited from object}}}

\begin{quote}
\_\_delattr\_\_(), \_\_format\_\_(), \_\_getattribute\_\_(), \_\_hash\_\_(), \_\_new\_\_(), \_\_reduce\_\_(), \_\_reduce\_ex\_\_(), \_\_repr\_\_(), \_\_setattr\_\_(), \_\_sizeof\_\_(), \_\_str\_\_(), \_\_subclasshook\_\_()
\end{quote}

%%%%%%%%%%%%%%%%%%%%%%%%%%%%%%%%%%%%%%%%%%%%%%%%%%%%%%%%%%%%%%%%%%%%%%%%%%%
%%                              Properties                               %%
%%%%%%%%%%%%%%%%%%%%%%%%%%%%%%%%%%%%%%%%%%%%%%%%%%%%%%%%%%%%%%%%%%%%%%%%%%%

  \subsubsection{Properties}

    \vspace{-1cm}
\hspace{\varindent}\begin{longtable}{|p{\varnamewidth}|p{\vardescrwidth}|l}
\cline{1-2}
\cline{1-2} \centering \textbf{Name} & \centering \textbf{Description}& \\
\cline{1-2}
\endhead\cline{1-2}\multicolumn{3}{r}{\small\textit{continued on next page}}\\\endfoot\cline{1-2}
\endlastfoot\multicolumn{2}{|l|}{\textit{Inherited from object}}\\
\multicolumn{2}{|p{\varwidth}|}{\raggedright \_\_class\_\_}\\
\cline{1-2}
\end{longtable}


%%%%%%%%%%%%%%%%%%%%%%%%%%%%%%%%%%%%%%%%%%%%%%%%%%%%%%%%%%%%%%%%%%%%%%%%%%%
%%                          Instance Variables                           %%
%%%%%%%%%%%%%%%%%%%%%%%%%%%%%%%%%%%%%%%%%%%%%%%%%%%%%%%%%%%%%%%%%%%%%%%%%%%

  \subsubsection{Instance Variables}

    \vspace{-1cm}
\hspace{\varindent}\begin{longtable}{|p{\varnamewidth}|p{\vardescrwidth}|l}
\cline{1-2}
\cline{1-2} \centering \textbf{Name} & \centering \textbf{Description}& \\
\cline{1-2}
\endhead\cline{1-2}\multicolumn{3}{r}{\small\textit{continued on next page}}\\\endfoot\cline{1-2}
\endlastfoot\raggedright p\-r\-o\-x\-y\-\_\-p\-o\-r\-t\- & \begin{alltt}
Initialize the DropboxClient object.

Args:
    session: A dropbox.session.DropboxSession object to use for making requests.
\end{alltt}&\\
\cline{1-2}
\end{longtable}

    \index{lib \textit{(package)}!lib.dropbox \textit{(module)}!lib.dropbox.DropboxClient \textit{(class)}|)}

%%%%%%%%%%%%%%%%%%%%%%%%%%%%%%%%%%%%%%%%%%%%%%%%%%%%%%%%%%%%%%%%%%%%%%%%%%%
%%                           Class Description                           %%
%%%%%%%%%%%%%%%%%%%%%%%%%%%%%%%%%%%%%%%%%%%%%%%%%%%%%%%%%%%%%%%%%%%%%%%%%%%

    \index{lib \textit{(package)}!lib.dropbox \textit{(module)}!lib.dropbox.RESTClient \textit{(class)}|(}
\subsection{Class RESTClient}

    \label{lib:dropbox:RESTClient}
\begin{tabular}{cccccc}
% Line for object, linespec=[False]
\multicolumn{2}{r}{\settowidth{\BCL}{object}\multirow{2}{\BCL}{object}}
&&
  \\\cline{3-3}
  &&\multicolumn{1}{c|}{}
&&
  \\
&&\multicolumn{2}{l}{\textbf{lib.dropbox.RESTClient}}
\end{tabular}

An class with all static methods to perform JSON REST requests that is used
internally by the Dropbox Client API. It provides just enough gear to make 
requests and get responses as JSON data (when applicable). All requests 
happen over SSL.


%%%%%%%%%%%%%%%%%%%%%%%%%%%%%%%%%%%%%%%%%%%%%%%%%%%%%%%%%%%%%%%%%%%%%%%%%%%
%%                                Methods                                %%
%%%%%%%%%%%%%%%%%%%%%%%%%%%%%%%%%%%%%%%%%%%%%%%%%%%%%%%%%%%%%%%%%%%%%%%%%%%

  \subsubsection{Methods}

    \label{lib:dropbox:RESTClient:request}
    \index{lib \textit{(package)}!lib.dropbox \textit{(module)}!lib.dropbox.RESTClient \textit{(class)}!lib.dropbox.RESTClient.request \textit{(static method)}}

    \vspace{0.5ex}

\hspace{.8\funcindent}\begin{boxedminipage}{\funcwidth}

    \raggedright \textbf{request}(\textit{method}, \textit{url}, \textit{post\_params}={\tt None}, \textit{body}={\tt None}, \textit{headers}={\tt None}, \textit{raw\_response}={\tt False}, \textit{proxy\_host}={\tt None}, \textit{proxy\_port}={\tt None})

    \vspace{-1.5ex}

    \rule{\textwidth}{0.5\fboxrule}
\setlength{\parskip}{2ex}
\begin{alltt}
Perform a REST request and parse the response.

Args:
    method: An HTTP method (e.g. 'GET' or 'POST').
    url: The URL to make a request to.
    post\_params: A dictionary of parameters to put in the body of the request.
        This option may not be used if the body parameter is given.
    body: The body of the request. Typically, this value will be a string.
        It may also be a file-like object in Python 2.6 and above. The body
        parameter may not be used with the post\_params parameter.
    headers: A dictionary of headers to send with the request.
    raw\_response: Whether to return the raw httplib.HTTPReponse object. [default False]
        It's best enabled for requests that return large amounts of data that you
        would want to .read() incrementally rather than loading into memory. Also
        use this for calls where you need to read metadata like status or headers,
        or if the body is not JSON.

Returns:
    The JSON-decoded data from the server, unless raw\_response is
    specified, in which case an httplib.HTTPReponse object is returned instead.

Raises:
    dropbox.rest.ErrorResponse: The returned HTTP status is not 200, or the body was
        not parsed from JSON successfully.
    dropbox.rest.RESTSocketError: A socket.error was raised while contacting Dropbox.
\end{alltt}

\setlength{\parskip}{1ex}
    \end{boxedminipage}

    \label{lib:dropbox:RESTClient:GET}
    \index{lib \textit{(package)}!lib.dropbox \textit{(module)}!lib.dropbox.RESTClient \textit{(class)}!lib.dropbox.RESTClient.GET \textit{(class method)}}

    \vspace{0.5ex}

\hspace{.8\funcindent}\begin{boxedminipage}{\funcwidth}

    \raggedright \textbf{GET}(\textit{cls}, \textit{url}, \textit{headers}={\tt None}, \textit{raw\_response}={\tt False})

    \vspace{-1.5ex}

    \rule{\textwidth}{0.5\fboxrule}
\setlength{\parskip}{2ex}
    Perform a GET request using RESTClient.request

\setlength{\parskip}{1ex}
    \end{boxedminipage}

    \label{lib:dropbox:RESTClient:POST}
    \index{lib \textit{(package)}!lib.dropbox \textit{(module)}!lib.dropbox.RESTClient \textit{(class)}!lib.dropbox.RESTClient.POST \textit{(class method)}}

    \vspace{0.5ex}

\hspace{.8\funcindent}\begin{boxedminipage}{\funcwidth}

    \raggedright \textbf{POST}(\textit{cls}, \textit{url}, \textit{params}={\tt None}, \textit{headers}={\tt None}, \textit{raw\_response}={\tt False})

    \vspace{-1.5ex}

    \rule{\textwidth}{0.5\fboxrule}
\setlength{\parskip}{2ex}
    Perform a POST request using RESTClient.request

\setlength{\parskip}{1ex}
    \end{boxedminipage}

    \label{lib:dropbox:RESTClient:PUT}
    \index{lib \textit{(package)}!lib.dropbox \textit{(module)}!lib.dropbox.RESTClient \textit{(class)}!lib.dropbox.RESTClient.PUT \textit{(class method)}}

    \vspace{0.5ex}

\hspace{.8\funcindent}\begin{boxedminipage}{\funcwidth}

    \raggedright \textbf{PUT}(\textit{cls}, \textit{url}, \textit{body}, \textit{headers}={\tt None}, \textit{raw\_response}={\tt False})

    \vspace{-1.5ex}

    \rule{\textwidth}{0.5\fboxrule}
\setlength{\parskip}{2ex}
    Perform a PUT request using RESTClient.request

\setlength{\parskip}{1ex}
    \end{boxedminipage}


\large{\textbf{\textit{Inherited from object}}}

\begin{quote}
\_\_delattr\_\_(), \_\_format\_\_(), \_\_getattribute\_\_(), \_\_hash\_\_(), \_\_init\_\_(), \_\_new\_\_(), \_\_reduce\_\_(), \_\_reduce\_ex\_\_(), \_\_repr\_\_(), \_\_setattr\_\_(), \_\_sizeof\_\_(), \_\_str\_\_(), \_\_subclasshook\_\_()
\end{quote}

%%%%%%%%%%%%%%%%%%%%%%%%%%%%%%%%%%%%%%%%%%%%%%%%%%%%%%%%%%%%%%%%%%%%%%%%%%%
%%                              Properties                               %%
%%%%%%%%%%%%%%%%%%%%%%%%%%%%%%%%%%%%%%%%%%%%%%%%%%%%%%%%%%%%%%%%%%%%%%%%%%%

  \subsubsection{Properties}

    \vspace{-1cm}
\hspace{\varindent}\begin{longtable}{|p{\varnamewidth}|p{\vardescrwidth}|l}
\cline{1-2}
\cline{1-2} \centering \textbf{Name} & \centering \textbf{Description}& \\
\cline{1-2}
\endhead\cline{1-2}\multicolumn{3}{r}{\small\textit{continued on next page}}\\\endfoot\cline{1-2}
\endlastfoot\multicolumn{2}{|l|}{\textit{Inherited from object}}\\
\multicolumn{2}{|p{\varwidth}|}{\raggedright \_\_class\_\_}\\
\cline{1-2}
\end{longtable}

    \index{lib \textit{(package)}!lib.dropbox \textit{(module)}!lib.dropbox.RESTClient \textit{(class)}|)}

%%%%%%%%%%%%%%%%%%%%%%%%%%%%%%%%%%%%%%%%%%%%%%%%%%%%%%%%%%%%%%%%%%%%%%%%%%%
%%                           Class Description                           %%
%%%%%%%%%%%%%%%%%%%%%%%%%%%%%%%%%%%%%%%%%%%%%%%%%%%%%%%%%%%%%%%%%%%%%%%%%%%

    \index{lib \textit{(package)}!lib.dropbox \textit{(module)}!lib.dropbox.RESTSocketError \textit{(class)}|(}
\subsection{Class RESTSocketError}

    \label{lib:dropbox:RESTSocketError}
\begin{tabular}{cccccccccccccccccc}
% Line for object, linespec=[False, False, False, False, False, False, False]
\multicolumn{2}{r}{\settowidth{\BCL}{object}\multirow{2}{\BCL}{object}}
&&
&&
&&
&&
&&
&&
&&
  \\\cline{3-3}
  &&\multicolumn{1}{c|}{}
&&
&&
&&
&&
&&
&&
&&
  \\
% Line for exceptions.BaseException, linespec=[False, False, False, False, False, False]
\multicolumn{4}{r}{\settowidth{\BCL}{exceptions.BaseException}\multirow{2}{\BCL}{exceptions.BaseException}}
&&
&&
&&
&&
&&
&&
  \\\cline{5-5}
  &&&&\multicolumn{1}{c|}{}
&&
&&
&&
&&
&&
&&
  \\
% Line for exceptions.Exception, linespec=[False, False, False, False, False]
\multicolumn{6}{r}{\settowidth{\BCL}{exceptions.Exception}\multirow{2}{\BCL}{exceptions.Exception}}
&&
&&
&&
&&
&&
  \\\cline{7-7}
  &&&&&&\multicolumn{1}{c|}{}
&&
&&
&&
&&
&&
  \\
% Line for exceptions.StandardError, linespec=[False, False, False, False]
\multicolumn{8}{r}{\settowidth{\BCL}{exceptions.StandardError}\multirow{2}{\BCL}{exceptions.StandardError}}
&&
&&
&&
&&
  \\\cline{9-9}
  &&&&&&&&\multicolumn{1}{c|}{}
&&
&&
&&
&&
  \\
% Line for exceptions.EnvironmentError, linespec=[False, False, False]
\multicolumn{10}{r}{\settowidth{\BCL}{exceptions.EnvironmentError}\multirow{2}{\BCL}{exceptions.EnvironmentError}}
&&
&&
&&
  \\\cline{11-11}
  &&&&&&&&&&\multicolumn{1}{c|}{}
&&
&&
&&
  \\
% Line for exceptions.IOError, linespec=[False, False]
\multicolumn{12}{r}{\settowidth{\BCL}{exceptions.IOError}\multirow{2}{\BCL}{exceptions.IOError}}
&&
&&
  \\\cline{13-13}
  &&&&&&&&&&&&\multicolumn{1}{c|}{}
&&
&&
  \\
% Line for socket.error, linespec=[False]
\multicolumn{14}{r}{\settowidth{\BCL}{socket.error}\multirow{2}{\BCL}{socket.error}}
&&
  \\\cline{15-15}
  &&&&&&&&&&&&&&\multicolumn{1}{c|}{}
&&
  \\
&&&&&&&&&&&&&&\multicolumn{2}{l}{\textbf{lib.dropbox.RESTSocketError}}
\end{tabular}

A light wrapper for socket.errors raised by dropbox.rest.RESTClient.request
that adds more information to the socket.error.


%%%%%%%%%%%%%%%%%%%%%%%%%%%%%%%%%%%%%%%%%%%%%%%%%%%%%%%%%%%%%%%%%%%%%%%%%%%
%%                                Methods                                %%
%%%%%%%%%%%%%%%%%%%%%%%%%%%%%%%%%%%%%%%%%%%%%%%%%%%%%%%%%%%%%%%%%%%%%%%%%%%

  \subsubsection{Methods}

    \vspace{0.5ex}

\hspace{.8\funcindent}\begin{boxedminipage}{\funcwidth}

    \raggedright \textbf{\_\_init\_\_}(\textit{self}, \textit{host}, \textit{e})

\setlength{\parskip}{2ex}
    x.\_\_init\_\_(...) initializes x; see help(type(x)) for signature

\setlength{\parskip}{1ex}
      Overrides: object.\_\_init\_\_ 	extit{(inherited documentation)}

    \end{boxedminipage}


\large{\textbf{\textit{Inherited from exceptions.IOError}}}

\begin{quote}
\_\_new\_\_()
\end{quote}

\large{\textbf{\textit{Inherited from exceptions.EnvironmentError}}}

\begin{quote}
\_\_reduce\_\_(), \_\_str\_\_()
\end{quote}

\large{\textbf{\textit{Inherited from exceptions.BaseException}}}

\begin{quote}
\_\_delattr\_\_(), \_\_getattribute\_\_(), \_\_getitem\_\_(), \_\_getslice\_\_(), \_\_repr\_\_(), \_\_setattr\_\_(), \_\_setstate\_\_(), \_\_unicode\_\_()
\end{quote}

\large{\textbf{\textit{Inherited from object}}}

\begin{quote}
\_\_format\_\_(), \_\_hash\_\_(), \_\_reduce\_ex\_\_(), \_\_sizeof\_\_(), \_\_subclasshook\_\_()
\end{quote}

%%%%%%%%%%%%%%%%%%%%%%%%%%%%%%%%%%%%%%%%%%%%%%%%%%%%%%%%%%%%%%%%%%%%%%%%%%%
%%                              Properties                               %%
%%%%%%%%%%%%%%%%%%%%%%%%%%%%%%%%%%%%%%%%%%%%%%%%%%%%%%%%%%%%%%%%%%%%%%%%%%%

  \subsubsection{Properties}

    \vspace{-1cm}
\hspace{\varindent}\begin{longtable}{|p{\varnamewidth}|p{\vardescrwidth}|l}
\cline{1-2}
\cline{1-2} \centering \textbf{Name} & \centering \textbf{Description}& \\
\cline{1-2}
\endhead\cline{1-2}\multicolumn{3}{r}{\small\textit{continued on next page}}\\\endfoot\cline{1-2}
\endlastfoot\multicolumn{2}{|l|}{\textit{Inherited from exceptions.EnvironmentError}}\\
\multicolumn{2}{|p{\varwidth}|}{\raggedright errno, filename, strerror}\\
\cline{1-2}
\multicolumn{2}{|l|}{\textit{Inherited from exceptions.BaseException}}\\
\multicolumn{2}{|p{\varwidth}|}{\raggedright args, message}\\
\cline{1-2}
\multicolumn{2}{|l|}{\textit{Inherited from object}}\\
\multicolumn{2}{|p{\varwidth}|}{\raggedright \_\_class\_\_}\\
\cline{1-2}
\end{longtable}

    \index{lib \textit{(package)}!lib.dropbox \textit{(module)}!lib.dropbox.RESTSocketError \textit{(class)}|)}

%%%%%%%%%%%%%%%%%%%%%%%%%%%%%%%%%%%%%%%%%%%%%%%%%%%%%%%%%%%%%%%%%%%%%%%%%%%
%%                           Class Description                           %%
%%%%%%%%%%%%%%%%%%%%%%%%%%%%%%%%%%%%%%%%%%%%%%%%%%%%%%%%%%%%%%%%%%%%%%%%%%%

    \index{lib \textit{(package)}!lib.dropbox \textit{(module)}!lib.dropbox.ErrorResponse \textit{(class)}|(}
\subsection{Class ErrorResponse}

    \label{lib:dropbox:ErrorResponse}
\begin{tabular}{cccccccccc}
% Line for object, linespec=[False, False, False]
\multicolumn{2}{r}{\settowidth{\BCL}{object}\multirow{2}{\BCL}{object}}
&&
&&
&&
  \\\cline{3-3}
  &&\multicolumn{1}{c|}{}
&&
&&
&&
  \\
% Line for exceptions.BaseException, linespec=[False, False]
\multicolumn{4}{r}{\settowidth{\BCL}{exceptions.BaseException}\multirow{2}{\BCL}{exceptions.BaseException}}
&&
&&
  \\\cline{5-5}
  &&&&\multicolumn{1}{c|}{}
&&
&&
  \\
% Line for exceptions.Exception, linespec=[False]
\multicolumn{6}{r}{\settowidth{\BCL}{exceptions.Exception}\multirow{2}{\BCL}{exceptions.Exception}}
&&
  \\\cline{7-7}
  &&&&&&\multicolumn{1}{c|}{}
&&
  \\
&&&&&&\multicolumn{2}{l}{\textbf{lib.dropbox.ErrorResponse}}
\end{tabular}

\begin{alltt}

Raised by dropbox.rest.RESTClient.request for requests that:
- Return a non-200 HTTP response, or
- Have a non-JSON response body, or
- Have a malformed/missing header in the response.

Most errors that Dropbox returns will have a error field that is unpacked and
placed on the ErrorResponse exception. In some situations, a user\_error field
will also come back. Messages under user\_error are worth showing to an end-user
of your app, while other errors are likely only useful for you as the developer.
\end{alltt}


%%%%%%%%%%%%%%%%%%%%%%%%%%%%%%%%%%%%%%%%%%%%%%%%%%%%%%%%%%%%%%%%%%%%%%%%%%%
%%                                Methods                                %%
%%%%%%%%%%%%%%%%%%%%%%%%%%%%%%%%%%%%%%%%%%%%%%%%%%%%%%%%%%%%%%%%%%%%%%%%%%%

  \subsubsection{Methods}

    \vspace{0.5ex}

\hspace{.8\funcindent}\begin{boxedminipage}{\funcwidth}

    \raggedright \textbf{\_\_init\_\_}(\textit{self}, \textit{http\_resp})

\setlength{\parskip}{2ex}
    x.\_\_init\_\_(...) initializes x; see help(type(x)) for signature

\setlength{\parskip}{1ex}
      Overrides: object.\_\_init\_\_ 	extit{(inherited documentation)}

    \end{boxedminipage}

    \vspace{0.5ex}

\hspace{.8\funcindent}\begin{boxedminipage}{\funcwidth}

    \raggedright \textbf{\_\_str\_\_}(\textit{self})

\setlength{\parskip}{2ex}
    str(x)

\setlength{\parskip}{1ex}
      Overrides: object.\_\_str\_\_ 	extit{(inherited documentation)}

    \end{boxedminipage}


\large{\textbf{\textit{Inherited from exceptions.Exception}}}

\begin{quote}
\_\_new\_\_()
\end{quote}

\large{\textbf{\textit{Inherited from exceptions.BaseException}}}

\begin{quote}
\_\_delattr\_\_(), \_\_getattribute\_\_(), \_\_getitem\_\_(), \_\_getslice\_\_(), \_\_reduce\_\_(), \_\_repr\_\_(), \_\_setattr\_\_(), \_\_setstate\_\_(), \_\_unicode\_\_()
\end{quote}

\large{\textbf{\textit{Inherited from object}}}

\begin{quote}
\_\_format\_\_(), \_\_hash\_\_(), \_\_reduce\_ex\_\_(), \_\_sizeof\_\_(), \_\_subclasshook\_\_()
\end{quote}

%%%%%%%%%%%%%%%%%%%%%%%%%%%%%%%%%%%%%%%%%%%%%%%%%%%%%%%%%%%%%%%%%%%%%%%%%%%
%%                              Properties                               %%
%%%%%%%%%%%%%%%%%%%%%%%%%%%%%%%%%%%%%%%%%%%%%%%%%%%%%%%%%%%%%%%%%%%%%%%%%%%

  \subsubsection{Properties}

    \vspace{-1cm}
\hspace{\varindent}\begin{longtable}{|p{\varnamewidth}|p{\vardescrwidth}|l}
\cline{1-2}
\cline{1-2} \centering \textbf{Name} & \centering \textbf{Description}& \\
\cline{1-2}
\endhead\cline{1-2}\multicolumn{3}{r}{\small\textit{continued on next page}}\\\endfoot\cline{1-2}
\endlastfoot\multicolumn{2}{|l|}{\textit{Inherited from exceptions.BaseException}}\\
\multicolumn{2}{|p{\varwidth}|}{\raggedright args, message}\\
\cline{1-2}
\multicolumn{2}{|l|}{\textit{Inherited from object}}\\
\multicolumn{2}{|p{\varwidth}|}{\raggedright \_\_class\_\_}\\
\cline{1-2}
\end{longtable}

    \index{lib \textit{(package)}!lib.dropbox \textit{(module)}!lib.dropbox.ErrorResponse \textit{(class)}|)}

%%%%%%%%%%%%%%%%%%%%%%%%%%%%%%%%%%%%%%%%%%%%%%%%%%%%%%%%%%%%%%%%%%%%%%%%%%%
%%                           Class Description                           %%
%%%%%%%%%%%%%%%%%%%%%%%%%%%%%%%%%%%%%%%%%%%%%%%%%%%%%%%%%%%%%%%%%%%%%%%%%%%

    \index{lib \textit{(package)}!lib.dropbox \textit{(module)}!lib.dropbox.ProperHTTPSConnection \textit{(class)}|(}
\subsection{Class ProperHTTPSConnection}

    \label{lib:dropbox:ProperHTTPSConnection}
\begin{tabular}{cccccc}
% Line for httplib.HTTPConnection, linespec=[False]
\multicolumn{2}{r}{\settowidth{\BCL}{httplib.HTTPConnection}\multirow{2}{\BCL}{httplib.HTTPConnection}}
&&
  \\\cline{3-3}
  &&\multicolumn{1}{c|}{}
&&
  \\
&&\multicolumn{2}{l}{\textbf{lib.dropbox.ProperHTTPSConnection}}
\end{tabular}

httplib.HTTPSConnection is broken because it doesn't do server certificate 
validation.  This class does certificate validation by ensuring:

\begin{enumerate}

\setlength{\parskip}{0.5ex}
  \item The certificate sent down by the server has a signature chain to one of
    the certs in our 'trusted-certs.crt' (this is mostly handled by the 
    'ssl' module).

  \item The hostname in the certificate matches the hostname we're connecting 
    to.

\end{enumerate}


%%%%%%%%%%%%%%%%%%%%%%%%%%%%%%%%%%%%%%%%%%%%%%%%%%%%%%%%%%%%%%%%%%%%%%%%%%%
%%                                Methods                                %%
%%%%%%%%%%%%%%%%%%%%%%%%%%%%%%%%%%%%%%%%%%%%%%%%%%%%%%%%%%%%%%%%%%%%%%%%%%%

  \subsubsection{Methods}

    \vspace{0.5ex}

\hspace{.8\funcindent}\begin{boxedminipage}{\funcwidth}

    \raggedright \textbf{\_\_init\_\_}(\textit{self}, \textit{host}, \textit{port}, \textit{proxy\_host}={\tt None}, \textit{proxy\_port}={\tt None})

\setlength{\parskip}{2ex}
\setlength{\parskip}{1ex}
      Overrides: httplib.HTTPConnection.\_\_init\_\_

    \end{boxedminipage}

    \vspace{0.5ex}

\hspace{.8\funcindent}\begin{boxedminipage}{\funcwidth}

    \raggedright \textbf{connect}(\textit{self})

\setlength{\parskip}{2ex}
    Connect to the host and port specified in \_\_init\_\_.

\setlength{\parskip}{1ex}
      Overrides: httplib.HTTPConnection.connect 	extit{(inherited documentation)}

    \end{boxedminipage}


\large{\textbf{\textit{Inherited from httplib.HTTPConnection}}}

\begin{quote}
close(), endheaders(), getresponse(), putheader(), putrequest(), request(), send(), set\_debuglevel(), set\_tunnel()
\end{quote}

%%%%%%%%%%%%%%%%%%%%%%%%%%%%%%%%%%%%%%%%%%%%%%%%%%%%%%%%%%%%%%%%%%%%%%%%%%%
%%                            Class Variables                            %%
%%%%%%%%%%%%%%%%%%%%%%%%%%%%%%%%%%%%%%%%%%%%%%%%%%%%%%%%%%%%%%%%%%%%%%%%%%%

  \subsubsection{Class Variables}

    \vspace{-1cm}
\hspace{\varindent}\begin{longtable}{|p{\varnamewidth}|p{\vardescrwidth}|l}
\cline{1-2}
\cline{1-2} \centering \textbf{Name} & \centering \textbf{Description}& \\
\cline{1-2}
\endhead\cline{1-2}\multicolumn{3}{r}{\small\textit{continued on next page}}\\\endfoot\cline{1-2}
\endlastfoot\multicolumn{2}{|l|}{\textit{Inherited from httplib.HTTPConnection}}\\
\multicolumn{2}{|p{\varwidth}|}{\raggedright auto\_open, debuglevel, default\_port, strict}\\
\cline{1-2}
\end{longtable}

    \index{lib \textit{(package)}!lib.dropbox \textit{(module)}!lib.dropbox.ProperHTTPSConnection \textit{(class)}|)}

%%%%%%%%%%%%%%%%%%%%%%%%%%%%%%%%%%%%%%%%%%%%%%%%%%%%%%%%%%%%%%%%%%%%%%%%%%%
%%                           Class Description                           %%
%%%%%%%%%%%%%%%%%%%%%%%%%%%%%%%%%%%%%%%%%%%%%%%%%%%%%%%%%%%%%%%%%%%%%%%%%%%

    \index{lib \textit{(package)}!lib.dropbox \textit{(module)}!lib.dropbox.CertificateError \textit{(class)}|(}
\subsection{Class CertificateError}

    \label{lib:dropbox:CertificateError}
\begin{tabular}{cccccccccccccc}
% Line for object, linespec=[False, False, False, False, False]
\multicolumn{2}{r}{\settowidth{\BCL}{object}\multirow{2}{\BCL}{object}}
&&
&&
&&
&&
&&
  \\\cline{3-3}
  &&\multicolumn{1}{c|}{}
&&
&&
&&
&&
&&
  \\
% Line for exceptions.BaseException, linespec=[False, False, False, False]
\multicolumn{4}{r}{\settowidth{\BCL}{exceptions.BaseException}\multirow{2}{\BCL}{exceptions.BaseException}}
&&
&&
&&
&&
  \\\cline{5-5}
  &&&&\multicolumn{1}{c|}{}
&&
&&
&&
&&
  \\
% Line for exceptions.Exception, linespec=[False, False, False]
\multicolumn{6}{r}{\settowidth{\BCL}{exceptions.Exception}\multirow{2}{\BCL}{exceptions.Exception}}
&&
&&
&&
  \\\cline{7-7}
  &&&&&&\multicolumn{1}{c|}{}
&&
&&
&&
  \\
% Line for exceptions.StandardError, linespec=[False, False]
\multicolumn{8}{r}{\settowidth{\BCL}{exceptions.StandardError}\multirow{2}{\BCL}{exceptions.StandardError}}
&&
&&
  \\\cline{9-9}
  &&&&&&&&\multicolumn{1}{c|}{}
&&
&&
  \\
% Line for exceptions.ValueError, linespec=[False]
\multicolumn{10}{r}{\settowidth{\BCL}{exceptions.ValueError}\multirow{2}{\BCL}{exceptions.ValueError}}
&&
  \\\cline{11-11}
  &&&&&&&&&&\multicolumn{1}{c|}{}
&&
  \\
&&&&&&&&&&\multicolumn{2}{l}{\textbf{lib.dropbox.CertificateError}}
\end{tabular}


%%%%%%%%%%%%%%%%%%%%%%%%%%%%%%%%%%%%%%%%%%%%%%%%%%%%%%%%%%%%%%%%%%%%%%%%%%%
%%                                Methods                                %%
%%%%%%%%%%%%%%%%%%%%%%%%%%%%%%%%%%%%%%%%%%%%%%%%%%%%%%%%%%%%%%%%%%%%%%%%%%%

  \subsubsection{Methods}


\large{\textbf{\textit{Inherited from exceptions.ValueError}}}

\begin{quote}
\_\_init\_\_(), \_\_new\_\_()
\end{quote}

\large{\textbf{\textit{Inherited from exceptions.BaseException}}}

\begin{quote}
\_\_delattr\_\_(), \_\_getattribute\_\_(), \_\_getitem\_\_(), \_\_getslice\_\_(), \_\_reduce\_\_(), \_\_repr\_\_(), \_\_setattr\_\_(), \_\_setstate\_\_(), \_\_str\_\_(), \_\_unicode\_\_()
\end{quote}

\large{\textbf{\textit{Inherited from object}}}

\begin{quote}
\_\_format\_\_(), \_\_hash\_\_(), \_\_reduce\_ex\_\_(), \_\_sizeof\_\_(), \_\_subclasshook\_\_()
\end{quote}

%%%%%%%%%%%%%%%%%%%%%%%%%%%%%%%%%%%%%%%%%%%%%%%%%%%%%%%%%%%%%%%%%%%%%%%%%%%
%%                              Properties                               %%
%%%%%%%%%%%%%%%%%%%%%%%%%%%%%%%%%%%%%%%%%%%%%%%%%%%%%%%%%%%%%%%%%%%%%%%%%%%

  \subsubsection{Properties}

    \vspace{-1cm}
\hspace{\varindent}\begin{longtable}{|p{\varnamewidth}|p{\vardescrwidth}|l}
\cline{1-2}
\cline{1-2} \centering \textbf{Name} & \centering \textbf{Description}& \\
\cline{1-2}
\endhead\cline{1-2}\multicolumn{3}{r}{\small\textit{continued on next page}}\\\endfoot\cline{1-2}
\endlastfoot\multicolumn{2}{|l|}{\textit{Inherited from exceptions.BaseException}}\\
\multicolumn{2}{|p{\varwidth}|}{\raggedright args, message}\\
\cline{1-2}
\multicolumn{2}{|l|}{\textit{Inherited from object}}\\
\multicolumn{2}{|p{\varwidth}|}{\raggedright \_\_class\_\_}\\
\cline{1-2}
\end{longtable}

    \index{lib \textit{(package)}!lib.dropbox \textit{(module)}!lib.dropbox.CertificateError \textit{(class)}|)}

%%%%%%%%%%%%%%%%%%%%%%%%%%%%%%%%%%%%%%%%%%%%%%%%%%%%%%%%%%%%%%%%%%%%%%%%%%%
%%                           Class Description                           %%
%%%%%%%%%%%%%%%%%%%%%%%%%%%%%%%%%%%%%%%%%%%%%%%%%%%%%%%%%%%%%%%%%%%%%%%%%%%

    \index{lib \textit{(package)}!lib.dropbox \textit{(module)}!lib.dropbox.DropboxSession \textit{(class)}|(}
\subsection{Class DropboxSession}

    \label{lib:dropbox:DropboxSession}
\begin{tabular}{cccccc}
% Line for object, linespec=[False]
\multicolumn{2}{r}{\settowidth{\BCL}{object}\multirow{2}{\BCL}{object}}
&&
  \\\cline{3-3}
  &&\multicolumn{1}{c|}{}
&&
  \\
&&\multicolumn{2}{l}{\textbf{lib.dropbox.DropboxSession}}
\end{tabular}


%%%%%%%%%%%%%%%%%%%%%%%%%%%%%%%%%%%%%%%%%%%%%%%%%%%%%%%%%%%%%%%%%%%%%%%%%%%
%%                                Methods                                %%
%%%%%%%%%%%%%%%%%%%%%%%%%%%%%%%%%%%%%%%%%%%%%%%%%%%%%%%%%%%%%%%%%%%%%%%%%%%

  \subsubsection{Methods}

    \vspace{0.5ex}

\hspace{.8\funcindent}\begin{boxedminipage}{\funcwidth}

    \raggedright \textbf{\_\_init\_\_}(\textit{self}, \textit{consumer\_key}, \textit{consumer\_secret}, \textit{access\_type}, \textit{locale}={\tt None})

    \vspace{-1.5ex}

    \rule{\textwidth}{0.5\fboxrule}
\setlength{\parskip}{2ex}
\begin{alltt}
Initialize a DropboxSession object.

Your consumer key and secret are available
at https://www.dropbox.com/developers/apps

Args:
    access\_type: Either 'dropbox' or 'app\_folder'. All path-based operations
        will occur relative to either the user's Dropbox root directory
        or your application's app folder.
    locale: A locale string ('en', 'pt\_PT', etc.) [optional]
        The locale setting will be used to translate any user-facing error
        messages that the server generates. At this time Dropbox supports
        'en', 'es', 'fr', 'de', and 'ja', though we will be supporting more
        languages in the future. If you send a language the server doesn't
        support, messages will remain in English. Look for these translated
        messages in rest.ErrorResponse exceptions as e.user\_error\_msg.
\end{alltt}

\setlength{\parskip}{1ex}
      Overrides: object.\_\_init\_\_

    \end{boxedminipage}

    \label{lib:dropbox:DropboxSession:is_linked}
    \index{lib \textit{(package)}!lib.dropbox \textit{(module)}!lib.dropbox.DropboxSession \textit{(class)}!lib.dropbox.DropboxSession.is\_linked \textit{(method)}}

    \vspace{0.5ex}

\hspace{.8\funcindent}\begin{boxedminipage}{\funcwidth}

    \raggedright \textbf{is\_linked}(\textit{self})

    \vspace{-1.5ex}

    \rule{\textwidth}{0.5\fboxrule}
\setlength{\parskip}{2ex}
    Return whether the DropboxSession has an access token attached.

\setlength{\parskip}{1ex}
    \end{boxedminipage}

    \label{lib:dropbox:DropboxSession:unlink}
    \index{lib \textit{(package)}!lib.dropbox \textit{(module)}!lib.dropbox.DropboxSession \textit{(class)}!lib.dropbox.DropboxSession.unlink \textit{(method)}}

    \vspace{0.5ex}

\hspace{.8\funcindent}\begin{boxedminipage}{\funcwidth}

    \raggedright \textbf{unlink}(\textit{self})

    \vspace{-1.5ex}

    \rule{\textwidth}{0.5\fboxrule}
\setlength{\parskip}{2ex}
    Remove any attached access token from the DropboxSession.

\setlength{\parskip}{1ex}
    \end{boxedminipage}

    \label{lib:dropbox:DropboxSession:set_token}
    \index{lib \textit{(package)}!lib.dropbox \textit{(module)}!lib.dropbox.DropboxSession \textit{(class)}!lib.dropbox.DropboxSession.set\_token \textit{(method)}}

    \vspace{0.5ex}

\hspace{.8\funcindent}\begin{boxedminipage}{\funcwidth}

    \raggedright \textbf{set\_token}(\textit{self}, \textit{access\_token}, \textit{access\_token\_secret})

    \vspace{-1.5ex}

    \rule{\textwidth}{0.5\fboxrule}
\setlength{\parskip}{2ex}
    Attach an access token to the DropboxSession.

    Note that the access 'token' is made up of both a token string and a 
    secret string.

\setlength{\parskip}{1ex}
    \end{boxedminipage}

    \label{lib:dropbox:DropboxSession:set_request_token}
    \index{lib \textit{(package)}!lib.dropbox \textit{(module)}!lib.dropbox.DropboxSession \textit{(class)}!lib.dropbox.DropboxSession.set\_request\_token \textit{(method)}}

    \vspace{0.5ex}

\hspace{.8\funcindent}\begin{boxedminipage}{\funcwidth}

    \raggedright \textbf{set\_request\_token}(\textit{self}, \textit{request\_token}, \textit{request\_token\_secret})

    \vspace{-1.5ex}

    \rule{\textwidth}{0.5\fboxrule}
\setlength{\parskip}{2ex}
    Attach an request token to the DropboxSession.

    Note that the reuest 'token' is made up of both a token string and a 
    secret string.

\setlength{\parskip}{1ex}
    \end{boxedminipage}

    \label{lib:dropbox:DropboxSession:build_path}
    \index{lib \textit{(package)}!lib.dropbox \textit{(module)}!lib.dropbox.DropboxSession \textit{(class)}!lib.dropbox.DropboxSession.build\_path \textit{(method)}}

    \vspace{0.5ex}

\hspace{.8\funcindent}\begin{boxedminipage}{\funcwidth}

    \raggedright \textbf{build\_path}(\textit{self}, \textit{target}, \textit{params}={\tt None})

    \vspace{-1.5ex}

    \rule{\textwidth}{0.5\fboxrule}
\setlength{\parskip}{2ex}
\begin{alltt}
Build the path component for an API URL.

This method urlencodes the parameters, adds them
to the end of the target url, and puts a marker for the API
version in front.

Args:
    target: A target url (e.g. '/files') to build upon.
    params: A dictionary of parameters (name to value). [optional]

Returns:
    The path and parameters components of an API URL.
\end{alltt}

\setlength{\parskip}{1ex}
    \end{boxedminipage}

    \label{lib:dropbox:DropboxSession:build_url}
    \index{lib \textit{(package)}!lib.dropbox \textit{(module)}!lib.dropbox.DropboxSession \textit{(class)}!lib.dropbox.DropboxSession.build\_url \textit{(method)}}

    \vspace{0.5ex}

\hspace{.8\funcindent}\begin{boxedminipage}{\funcwidth}

    \raggedright \textbf{build\_url}(\textit{self}, \textit{host}, \textit{target}, \textit{params}={\tt None})

    \vspace{-1.5ex}

    \rule{\textwidth}{0.5\fboxrule}
\setlength{\parskip}{2ex}
\begin{alltt}
Build an API URL.

This method adds scheme and hostname to the path
returned from build\_path.

Args:
    target: A target url (e.g. '/files') to build upon.
    params: A dictionary of parameters (name to value). [optional]

Returns:
    The full API URL.
\end{alltt}

\setlength{\parskip}{1ex}
    \end{boxedminipage}

    \label{lib:dropbox:DropboxSession:build_authorize_url}
    \index{lib \textit{(package)}!lib.dropbox \textit{(module)}!lib.dropbox.DropboxSession \textit{(class)}!lib.dropbox.DropboxSession.build\_authorize\_url \textit{(method)}}

    \vspace{0.5ex}

\hspace{.8\funcindent}\begin{boxedminipage}{\funcwidth}

    \raggedright \textbf{build\_authorize\_url}(\textit{self}, \textit{request\_token}, \textit{oauth\_callback}={\tt None})

    \vspace{-1.5ex}

    \rule{\textwidth}{0.5\fboxrule}
\setlength{\parskip}{2ex}
\begin{alltt}
Build a request token authorization URL.

After obtaining a request token, you'll need to send the user to
the URL returned from this function so that they can confirm that
they want to connect their account to your app.

Args:
    request\_token: A request token from obtain\_request\_token.
    oauth\_callback: A url to redirect back to with the authorized
        request token.

Returns:
    An authorization for the given request token.
\end{alltt}

\setlength{\parskip}{1ex}
    \end{boxedminipage}

    \label{lib:dropbox:DropboxSession:obtain_request_token}
    \index{lib \textit{(package)}!lib.dropbox \textit{(module)}!lib.dropbox.DropboxSession \textit{(class)}!lib.dropbox.DropboxSession.obtain\_request\_token \textit{(method)}}

    \vspace{0.5ex}

\hspace{.8\funcindent}\begin{boxedminipage}{\funcwidth}

    \raggedright \textbf{obtain\_request\_token}(\textit{self})

    \vspace{-1.5ex}

    \rule{\textwidth}{0.5\fboxrule}
\setlength{\parskip}{2ex}
\begin{alltt}
Obtain a request token from the Dropbox API.

This is your first step in the OAuth process.  You call this to get a
request\_token from the Dropbox server that you can then use with
DropboxSession.build\_authorize\_url() to get the user to authorize it.
After it's authorized you use this token with
DropboxSession.obtain\_access\_token() to get an access token.

NOTE:  You should only need to do this once for each user, and then you
can store the access token for that user for later operations.

Returns:
    An oauth.OAuthToken representing the request token Dropbox assigned
    to this app. Also attaches the request token as self.request\_token.
\end{alltt}

\setlength{\parskip}{1ex}
    \end{boxedminipage}

    \label{lib:dropbox:DropboxSession:obtain_access_token}
    \index{lib \textit{(package)}!lib.dropbox \textit{(module)}!lib.dropbox.DropboxSession \textit{(class)}!lib.dropbox.DropboxSession.obtain\_access\_token \textit{(method)}}

    \vspace{0.5ex}

\hspace{.8\funcindent}\begin{boxedminipage}{\funcwidth}

    \raggedright \textbf{obtain\_access\_token}(\textit{self}, \textit{request\_token}={\tt None})

    \vspace{-1.5ex}

    \rule{\textwidth}{0.5\fboxrule}
\setlength{\parskip}{2ex}
\begin{alltt}
Obtain an access token for a user.

After you get a request token, and then send the user to the authorize
URL, you can use the authorized request token with this method to get the
access token to use for future operations. The access token is stored on
the session object.

Args:
    request\_token: A request token from obtain\_request\_token. [optional]
        The request\_token should have been authorized via the
        authorization url from build\_authorize\_url. If you don't pass
        a request\_token, the fallback is self.request\_token, which
        will exist if you previously called obtain\_request\_token on this
        DropboxSession instance.

Returns:
    An oauth.OAuthToken representing the access token Dropbox assigned
    to this app and user. Also attaches the access token as self.token.
\end{alltt}

\setlength{\parskip}{1ex}
    \end{boxedminipage}

    \label{lib:dropbox:DropboxSession:build_access_headers}
    \index{lib \textit{(package)}!lib.dropbox \textit{(module)}!lib.dropbox.DropboxSession \textit{(class)}!lib.dropbox.DropboxSession.build\_access\_headers \textit{(method)}}

    \vspace{0.5ex}

\hspace{.8\funcindent}\begin{boxedminipage}{\funcwidth}

    \raggedright \textbf{build\_access\_headers}(\textit{self}, \textit{method}, \textit{resource\_url}, \textit{params}={\tt None}, \textit{request\_token}={\tt None})

    \vspace{-1.5ex}

    \rule{\textwidth}{0.5\fboxrule}
\setlength{\parskip}{2ex}
\begin{alltt}
Build OAuth access headers for a future request.

Args:
    method: The HTTP method being used (e.g. 'GET' or 'POST').
    resource\_url: The full url the request will be made to.
    params: A dictionary of parameters to add to what's already on the url.
        Typically, this would consist of POST parameters.

Returns:
    A tuple of (header\_dict, params) where header\_dict is a dictionary
    of header names and values appropriate for passing into dropbox.rest.RESTClient
    and params is a dictionary like the one that was passed in, but augmented with
    oauth-related parameters as appropriate.
\end{alltt}

\setlength{\parskip}{1ex}
    \end{boxedminipage}


\large{\textbf{\textit{Inherited from object}}}

\begin{quote}
\_\_delattr\_\_(), \_\_format\_\_(), \_\_getattribute\_\_(), \_\_hash\_\_(), \_\_new\_\_(), \_\_reduce\_\_(), \_\_reduce\_ex\_\_(), \_\_repr\_\_(), \_\_setattr\_\_(), \_\_sizeof\_\_(), \_\_str\_\_(), \_\_subclasshook\_\_()
\end{quote}

%%%%%%%%%%%%%%%%%%%%%%%%%%%%%%%%%%%%%%%%%%%%%%%%%%%%%%%%%%%%%%%%%%%%%%%%%%%
%%                              Properties                               %%
%%%%%%%%%%%%%%%%%%%%%%%%%%%%%%%%%%%%%%%%%%%%%%%%%%%%%%%%%%%%%%%%%%%%%%%%%%%

  \subsubsection{Properties}

    \vspace{-1cm}
\hspace{\varindent}\begin{longtable}{|p{\varnamewidth}|p{\vardescrwidth}|l}
\cline{1-2}
\cline{1-2} \centering \textbf{Name} & \centering \textbf{Description}& \\
\cline{1-2}
\endhead\cline{1-2}\multicolumn{3}{r}{\small\textit{continued on next page}}\\\endfoot\cline{1-2}
\endlastfoot\multicolumn{2}{|l|}{\textit{Inherited from object}}\\
\multicolumn{2}{|p{\varwidth}|}{\raggedright \_\_class\_\_}\\
\cline{1-2}
\end{longtable}


%%%%%%%%%%%%%%%%%%%%%%%%%%%%%%%%%%%%%%%%%%%%%%%%%%%%%%%%%%%%%%%%%%%%%%%%%%%
%%                            Class Variables                            %%
%%%%%%%%%%%%%%%%%%%%%%%%%%%%%%%%%%%%%%%%%%%%%%%%%%%%%%%%%%%%%%%%%%%%%%%%%%%

  \subsubsection{Class Variables}

    \vspace{-1cm}
\hspace{\varindent}\begin{longtable}{|p{\varnamewidth}|p{\vardescrwidth}|l}
\cline{1-2}
\cline{1-2} \centering \textbf{Name} & \centering \textbf{Description}& \\
\cline{1-2}
\endhead\cline{1-2}\multicolumn{3}{r}{\small\textit{continued on next page}}\\\endfoot\cline{1-2}
\endlastfoot\raggedright A\-P\-I\-\_\-V\-E\-R\-S\-I\-O\-N\- & \raggedright \textbf{Value:} 
{\tt 1}&\\
\cline{1-2}
\raggedright A\-P\-I\-\_\-H\-O\-S\-T\- & \raggedright \textbf{Value:} 
{\tt \texttt{'}\texttt{api.dropbox.com}\texttt{'}}&\\
\cline{1-2}
\raggedright W\-E\-B\-\_\-H\-O\-S\-T\- & \raggedright \textbf{Value:} 
{\tt \texttt{'}\texttt{www.dropbox.com}\texttt{'}}&\\
\cline{1-2}
\raggedright A\-P\-I\-\_\-C\-O\-N\-T\-E\-N\-T\-\_\-H\-O\-S\-T\- & \raggedright \textbf{Value:} 
{\tt \texttt{'}\texttt{api-content.dropbox.com}\texttt{'}}&\\
\cline{1-2}
\end{longtable}

    \index{lib \textit{(package)}!lib.dropbox \textit{(module)}!lib.dropbox.DropboxSession \textit{(class)}|)}
    \index{lib \textit{(package)}!lib.dropbox \textit{(module)}|)}
